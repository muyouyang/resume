% !TEX TS-program = xelatex
% !TEX encoding = UTF-8 Unicode
% !Mode:: "TeX:UTF-8"
\documentclass[14pt]{resume}
\usepackage{graphicx}
\usepackage{tabu}
\usepackage{multirow}
\usepackage{multicol}
\usepackage{progressbar}
\usepackage{zh_CN-Adobefonts_external}
\usepackage{linespacing_fix}
\usepackage{cite}

\begin{document}
\pagenumbering{gobble}

\begin{multicols}{4}
    \Large{
        \begin{tabu}{ r }
            \multirow{5}{1in}{
                \includegraphics[width=0.88in]{聂韵致证件照.png}
            }
        \end{tabu}
    }
    \columnbreak
    \Large{
        \begin{tabu}{ l l }
            ~\\
            & \faBirthdayCake{1999.01.05} \\
            & \phone{15895918878} \\
            & \email{m15895918878@163.com} \\
            \end{tabu}
    }
    \columnbreak
    \Large{
        \begin{tabu}{ r }
            \multirow{5}{3.5in}{
                \name{聂韵致}
                \basicInfo{
                    意向职位:视觉算法岗
                }
            }
        \end{tabu}
    }
\end{multicols}


% 教育背景
\section{\faGraduationCap\  教育背景}
\datedsubsection{\textbf{东南大学(硕士)\quad\quad\quad}{ 电子信息工程(专硕) \quad\quad\quad }}{2018.09 - 2021.06}
\datedsubsection{\textbf{东南大学(本科)\quad\quad\quad}{ 信息工程     \quad\quad\quad\quad\quad}}{2014.09 - 2018.06}
\textbf{2014年15岁入学东南大学;本科期间成绩优异,免试保送本专业研究生}
% 实习经历
\section{\faBriefcase\ 实习经历}
\datedsubsection{\textbf{驭势科技\quad\quad}{南京AI研究院\quad\quad\quad\quad}{ 计算机视觉算法岗\quad\quad\quad}{2019.04 - 2018.09}}
\datedsubsubsection{\textbf{项目一名称:车道线检测(在图森数据集和公司内部数据集上使用line-cnn模型进行车道线检测)\quad\quad }}
\textbf{主要工作}
\begin{itemize}
    \item 对公司车道线数据集处理,本人实现包括纠正错误车道线标注类型,重叠线,断线的合并,并对车道线数量类型曲度做统计分析可视化功能;
    \item 本人与另一同伴搭建line-cnn框架,该框架在在论文《Line-CNN end to end traffic line detection with LPU》中提出,论文并未给出开源代码。
    本人主要负责特征提取模型选取编写以及设置anchor函数,左右下三边界anchor水平方向偏移量,有效偏移量计算函数等,以及主训练框架搭建,以及模型搭建完成后训练测试修正过程。
\end{itemize}
\textbf{实验改进}
\begin{itemize}
    \item 论文中对图森数据集准确率和FN指标为96.87\%和1.98\%,复现实际达到93.42\%和5.81\%,通过补全起始点坐标和增加anchor 类型准确率和FN到95.7\%和2.93\%基本达到论文要求。
\end{itemize}

\datedsubsubsection{\textbf{项目二名称:通用2D目标检测研究\quad\quad}}\\
\textbf{主要工作}
\begin{itemize}
    \item 深入研读FasterRCNN,SSD算法代码。
    \item 在公司数据集上研究对比了端到端训练,以及四步交替训练过程区别和结果分析,并同时研究对比了深入理解rpn模块和FasterRCNN检测模块代码,清楚掌握roi和nms模块。\item 基本复现SSD 代码。(在史建波导师指导下对目标检测基础算法进行理论研究)
\end{itemize}

% 项目经历
\section{\faUsers\ 项目经历}
\textbf{年报文本分类(实验室)}
\begin{itemize}
    \item 使用传统机器学习算法对公司年报完成行业分类
\end{itemize}

\textbf{基于LightGBM的客户信用风险预测(西南财经大学 新网银行杯 数据科学竞赛三等奖2018}
\begin{itemize}
    \item 根据提供的用户画像(id,行为数据等均为脱敏数据)使用LightGBM进行客户风险预测
\end{itemize}

\textbf{遥感目标检测(实验室)}
\begin{itemize}
    \item 针对遥感图像使用剪枝后的yolov3进行小目标检测提升准确率和flops
\end{itemize}

% 个人技能
\section{\faCogs\ 技能}
\begin{itemize}
    \item 熟练掌握c++,python,熟练使用深度学习库pytorch, 以及熟练使用mysql数据分析处理
    \item 英语雅思7
    
\end{itemize}
% 个人评价
\section{\faUniversity\ 个人评价}
本人自学能力极强,大学前跳2级。本科通信工程培养我清楚理解信息传输及改进需求,信息安全竞赛为我提供实践平台;文本分析项目第一次接触传统机器学习和深度学习算法,自学西瓜书和实践网上项目之后获取驭势科技南京研究院实习机会;实习中十分幸运在宾大教授史建波老师的指导下,从入门计算机视觉到深入了解目标检测领域算法,锻炼代码能力,培养解决问题的思考角度。本人仍欠缺实际落地工程项目经历,希望可以在贵公司实习参与到项目产品过程中,根据实际需求具体优化检测过程中的每个步骤。
\end{document}
© 2020 GitHub, Inc.
Terms
Privacy
Security
Status
Help
Contact GitHub
Pricing
API
Training
Blog
About
